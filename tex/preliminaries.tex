\section{Diagonal simplicial tensor modules}
\label{sec:dstm}

Throughout, let $A$ be a commutative ring and write $\mathbf{Mod}_A$ for the category of $A$-modules.
Denote by $\mathbb{N}$ the set of nonnegative integers and by $\mathbb{Z}^+$ the set of positive integers.

\subsection{Simplicial preliminaries}

The \textbf{simplicial category} $\Delta$ has objects $[p]:=\{0,\ldots,p\}$ for $p\in\mathbb{N}$; morphisms are nondecreasing maps.
It is generated by the \textbf{coface maps} $\delta_i^p:[p-1]\to[p]$ (the injection missing $i$) and the \textbf{codegeneracy maps} $\sigma_i^p:[p+1]\to[p]$ (the surjection repeating $i$).

A \textbf{simplicial $A$-module} is a functor $X_\bullet:\Delta^{\mathrm{op}}\to\mathbf{Mod}_A$.
We write $X_p:=X([p])$. The induced maps are the \textbf{face maps} $d_i^p:=X(\delta_i^p):X_p\to X_{p-1}$ and the \textbf{degeneracy maps} $s_i^p:=X(\sigma_i^p):X_p\to X_{p+1}$.
The associated chain complex $(X_\bullet, \partial_\bullet)$ has boundary maps $\partial_p = \sum_{i=0}^p (-1)^i d_i^p$.

\subsection{Setup and index sets}

We now define the index sets underlying the diagonal simplicial tensor module.
Let $k\in\mathbb{Z}^+$. We analyze modules consisting of tensors of order $k$.
A \textbf{shape} is a $k$-tuple $\vec{s}=(n_1,\dots,n_k)\in(\mathbb{Z}^+)^k$.

We fix the shape $\vec{s}$ throughout and define its \textbf{simplicial dimension} as
\[
n=n(\vec{s}):=\min(n_1,\dots,n_k)-1.
\]

For any degree $p\ge 0$ and axis $1\le a\le k$, we define the index bounds:
\[
M_a(p):=n_a-1-n+p.
\]
Since $n_a-1 \ge n$, we have $M_a(p)\ge p$ for all $a$. The index set in degree $p$ is defined as
\[
I_p:=\prod_{a=1}^k [\,M_a(p)\,].
\]
Since $[p]\subseteq[\,M_a(p)\,]$ for every $a$, we have the inclusion $[p]^k\subseteq I_p$.

\subsection{\texorpdfstring{The Cosimplicial Index Set $I_\bullet$}{The Cosimplicial Index Set I-bullet}}

We define a functor $I_\bullet: \Delta \to \mathbf{Set}$. On objects, $I_\bullet([p]) := I_p$. Note that $M_a(p\pm 1) = M_a(p)\pm 1$.

The structure maps are defined "diagonally" as products of the standard generators.

\begin{definition}[Cosimplicial Structure Maps]
The functor $I_\bullet$ maps the generators of $\Delta$ as follows:

For $p\ge 1$ and $0\le i\le p$, the \textbf{index coface map} $\Delta_i^p: I_{p-1} \to I_p$ is
\[
\Delta_i^p := I_\bullet(\delta_i^p) := \prod_{a=1}^k \delta_i^{M_a(p)}.
\]
For $p\ge 0$ and $0\le i\le p$, the \textbf{index codegeneracy map} $\Sigma_i^p: I_{p+1} \to I_p$ is
\[
\Sigma_i^p := I_\bullet(\sigma_i^p) := \prod_{a=1}^k \sigma_i^{M_a(p)}.
\]
\end{definition}

Since $M_a(p)\ge p$, all factors are defined for the common index $i$. Because the standard maps $\delta_i, \sigma_i$ satisfy the cosimplicial identities, the product maps $\Delta_i^p, \Sigma_i^p$ also satisfy them componentwise, proving that $I_\bullet$ is a cosimplicial set.

\subsection{\texorpdfstring{The Diagonal Simplicial Module $X_\bullet(\vec{s};A)$}{The Diagonal Simplicial Module}}

We define the contravariant functor $A^{(-)}: \mathbf{Set} \to \mathbf{Mod}_A$. This functor maps a set $S$ to the $A$-module of functions $S\to A$, and a map of sets $f: S_1 \to S_2$ to the module homomorphism $A^f: A^{S_2} \to A^{S_1}$ defined by pre-composition ($T \mapsto T\circ f$).

\begin{definition}[Diagonal simplicial tensor module]
The \textbf{diagonal simplicial $A$-tensor module} $X_\bullet(\vec{s};A):\Delta^{\mathrm{op}}\to\mathbf{Mod}_A$ is the composition $A^{(-)} \circ I_\bullet$.

On objects, $X_p(\vec{s};A):=A^{I_p}$. Since the index sets $I_p$ are finite, $X_p$ is a free $A$-module.

The face and degeneracy maps are induced by the corresponding maps in $I_\bullet$ via pre-composition:
\begin{align*}
d_i^p(T) &:= T\circ \Delta_i^p, \\
s_i^p(T) &:= T\circ \Sigma_i^p.
\end{align*}
\end{definition}

\begin{remark}[Matrices in the constant shape case]\label{rem:matrices-constant-shape}
If $k=2$ and $\vec s=(N,N)$ is constant, then $n=N-1$ and
\[
M_a(p) = n_a-1-n+p = p \qquad (a=1,2).
\]
Hence
\[
I_p = [p]^2
\qquad\text{and}\qquad
X_p(\vec s;A) = A^{[p]^2} \cong A^{(p+1)\times(p+1)}.
\]
In particular,
\[
X_{N-1}(\vec s;A) \cong A^{[N-1]^2} \cong A^{N\times N},
\]
so an $N\times N$ matrix may be viewed as a simplex in degree $N-1$ of $X_\bullet(\vec s;A)$.
This identification will be used when we regard adjacency matrices of graphs on $N$ vertices as elements of $X_{N-1}(\vec s;A)$.
\end{remark}


\subsection{Axis Symmetries and Permutations of Tensor Factors}

For a shape $\vec s=(n_1,\dots,n_k)$, define its stabilizer
\[
\operatorname{Stab}(\vec s)
 := \{\sigma\in S_k : n_{\sigma(a)} = n_a\ \text{for all }a\}.
\]
Each $\sigma\in\operatorname{Stab}(\vec s)$ acts (on the left) on the index sets $I_p$ by
permuting coordinates:
\[
\sigma\cdot (m_1,\dots,m_k)
 := (m_{\sigma^{-1}(1)},\dots,m_{\sigma^{-1}(k)}).
\]
This induces a (left) action on the diagonal simplicial module
$X_\bullet(\vec s;A)$ by precomposition:
\[
(\sigma\cdot T)(m) := T(\sigma^{-1}\cdot m),
 \qquad T\in X_p(\vec s;A),\ m\in I_p.
\]

\begin{lemma}[Equivariance under $\operatorname{Stab}(\vec s)$]\label{lem:equivariance-stab}
The action of $\operatorname{Stab}(\vec s)$ on $I_\bullet$ commutes with the cosimplicial structure maps. Consequently, for every $\sigma\in\operatorname{Stab}(\vec s)$ and all $p,i$ we have
\[
d_i^p(\sigma\cdot T) = \sigma\cdot d_i^p(T),
 \qquad
s_i^p(\sigma\cdot T) = \sigma\cdot s_i^p(T).
\]
The boundary operators $\partial_p$ also commute with the $\operatorname{Stab}(\vec s)$-action.
\end{lemma}

\begin{proof}
Let $\sigma \in \operatorname{Stab}(\vec{s})$. By definition, $n_a = n_{\sigma(a)}$, which implies $n_a = n_{\sigma^{-1}(a)}$. This ensures $M_a(p) = M_{\sigma^{-1}(a)}(p)$ for all $a, p$.

We first verify the commutation on the index sets. Let $m \in I_{p-1}$. We show $\sigma \cdot \Delta_i^p(m) = \Delta_i^p(\sigma \cdot m)$. The $a$-th component of the left side is
\[
(\sigma \cdot \Delta_i^p(m))_a = (\Delta_i^p(m))_{\sigma^{-1}(a)} = \delta_i^{M_{\sigma^{-1}(a)}(p)}(m_{\sigma^{-1}(a)}).
\]
The $a$-th component of the right side is
\[
(\Delta_i^p(\sigma \cdot m))_a = \delta_i^{M_a(p)}((\sigma \cdot m)_a) = \delta_i^{M_a(p)}(m_{\sigma^{-1}(a)}).
\]
Since $M_a(p) = M_{\sigma^{-1}(a)}(p)$, the components are equal. Thus $\sigma \circ \Delta_i^p = \Delta_i^p \circ \sigma$. The argument for $\Sigma_i^p$ is analogous.

Now we verify the equivariance on $X_p$. Let $T \in X_p$.
\begin{align*}
d_i^p(\sigma\cdot T) &= (\sigma\cdot T) \circ \Delta_i^p \\
 &= (T \circ \sigma^{-1}) \circ \Delta_i^p \\
 &= T \circ (\sigma^{-1} \circ \Delta_i^p) \\
 &= T \circ (\Delta_i^p \circ \sigma^{-1}) \quad (\text{by commutativity on } I_\bullet)\\
 &= (T \circ \Delta_i^p) \circ \sigma^{-1} \\
 &= \sigma \cdot (T \circ \Delta_i^p) = \sigma \cdot d_i^p(T).
\end{align*}
The proof for $s_i^p$ is analogous. The statement for $\partial_p$ follows by linearity.
\end{proof}

\begin{remark}[Constant shape and symmetry subcomplexes]
Suppose $\vec s=(N,\dots,N)$ is constant. Then
$\operatorname{Stab}(\vec s)\cong S_k$ acts on $X_\bullet(\vec s;A)$ by
permuting tensor axes. By Lemma~\ref{lem:equivariance-stab}, any $S_k$-stable subspace in each degree $X_p$ defines a simplicial $A$-submodule.

In particular, the symmetric and alternating subspaces
\[
X_\bullet^{\mathrm{Sym}}
 := \{T : \sigma\cdot T = T\ \forall\sigma\in S_k\},
 \qquad
X_\bullet^{\mathrm{Alt}}
 := \{T : \sigma\cdot T = \operatorname{sgn}(\sigma)\,T\}
\]
form simplicial $A$-submodules. (If $k!$ is invertible in $A$, these are direct summands.) In
\ref{app:filtration} we show that the global contracting homotopy
$H$ is $\operatorname{Stab}(\vec s)$-equivariant, so it restricts to
chain contractions on $X_\bullet^{\mathrm{Sym}}$ and $X_\bullet^{\mathrm{Alt}}$.
Hence these symmetry subcomplexes are also contractible.
\end{remark}

\begin{remark}[Matrix and Hermitian transpose]
For $k=2$ and $\vec s=(N,N)$, the nontrivial element $\tau \in S_2$ acts by swapping the tensor axes, corresponding to the transpose: $(\tau\cdot T)(m_1, m_2) = T(m_2, m_1)$. By Lemma~\ref{lem:equivariance-stab}, all structure maps and the boundary operators $\partial_p$ commute with this action.

Over $K=\mathbb C$, viewing $X_\bullet(\vec s;K)$ as a simplicial
$\mathbb R$-module, complex conjugation commutes with all structure maps. The Hermitian conjugate action (transpose combined with conjugation) also commutes with $d_i^p,s_i^p$ and $\partial_p$ (as $\mathbb R$-linear operators). The
real subspaces of symmetric, skew-symmetric, Hermitian, and skew-Hermitian
tensors form simplicial $\mathbb R$-submodules, and the global
contracting homotopy $H$ restricts to each of them.
\end{remark}

\subsection{Base Change and Realization Maps}

\begin{proposition}[Base change for $X_\bullet(\vec{s};-)$]\label{prop:base-change}
For any ring homomorphism $\varphi: A \to B$ and shape $\vec{s}$, there is a canonical isomorphism of simplicial $B$-modules
\[
X_\bullet(\vec{s};A) \otimes_A B \;\xrightarrow[\cong]{\theta_\bullet}\; X_\bullet(\vec{s};B).
\]
\end{proposition}

\begin{proof}
Each $X_p(\vec{s};A) = A^{I_p}$ is a finite free $A$-module with basis $\{E_m\}_{m \in I_p}$ (indicator functions). Define
\[
\theta_p: X_p(\vec{s};A) \otimes_A B \longrightarrow X_p(\vec{s};B), \qquad \theta_p(E_m \otimes b) = b \cdot E_m.
\]
Since $I_p$ is finite, $(A^{I_p}) \otimes_A B \cong B^{I_p}$, so each $\theta_p$ is a $B$-module isomorphism.

Simplicial compatibility follows because the face and degeneracy maps
are given by precomposition with the same maps
$\Delta_i^p,\Sigma_i^p$ on $I_\bullet$ in both
$X_\bullet(\vec{s};A)$ and $X_\bullet(\vec{s};B)$, so each $\theta_p$
commutes with all $d_i^p$ and $s_i^p$.
\end{proof}

\begin{corollary}[Naturality and flat base change]\label{cor:base-change-naturality}
Let $T \in X_n(\vec{s};A)$, and let $f_T: A[\Delta^n] \to X_\bullet(\vec{s};A)$ be the corresponding realization map (defined in Section~\ref{sec:generated}). Let $\varphi: A \to B$ be a ring homomorphism, and let $T_B := \theta_n(T \otimes 1) \in X_n(\vec{s};B)$. Let $\varphi_p: A[\Delta^n]_p \otimes_A B \xrightarrow{\cong} B[\Delta^n]_p$ denote the canonical isomorphism.

\begin{enumerate}
    \item (Naturality) The following diagram commutes for all $p$:
\[
\begin{tikzcd}
A[\Delta^n]_p \otimes_A B \arrow[r, "f_{T,p} \otimes 1_B"] \arrow[d, "\varphi_p"', "\cong"] & X_p(\vec{s};A) \otimes_A B \arrow[d, "\theta_p"', "\cong"] \\
B[\Delta^n]_p \arrow[r, "f_{T_B,p}"'] & X_p(\vec{s};B)
\end{tikzcd}
\]
    \item (Kernel preservation) If $B$ is a flat $A$-module, the kernel sequence is preserved:
\[
(\ker f_{T,p}) \otimes_A B \;\cong\; \ker(f_{T_B,p}) \qquad (\forall p).
\]
\end{enumerate}
\end{corollary}

\begin{proof}
(1) Let $\alpha \in \Delta([p],[n])$ be a basis element of $A[\Delta^n]_p$. Both paths in the diagram map the element $\alpha \otimes b$ to $b \cdot f_{T_B,p}(\alpha)$, establishing commutativity.

(2) Consider the exact sequence $0 \to \ker f_{T,p} \to A[\Delta^n]_p \xrightarrow{f_{T,p}} X_p(\vec{s};A)$. If $B$ is a flat $A$-module, the functor $(-) \otimes_A B$ is exact. Applying it yields the exact sequence
\[
0 \to (\ker f_{T,p}) \otimes_A B \to A[\Delta^n]_p \otimes_A B \xrightarrow{f_{T,p} \otimes 1_B} X_p(\vec{s};A) \otimes_A B.
\]
This implies $(\ker f_{T,p}) \otimes_A B \cong \ker(f_{T,p} \otimes 1_B)$. By the commutativity of the diagram in (1) and the fact that $\varphi_p$ and $\theta_p$ are isomorphisms, $\ker(f_{T,p} \otimes 1_B)$ is isomorphic to $\ker(f_{T_B,p})$.
\end{proof}
