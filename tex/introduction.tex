\section{Introduction}
\label{sec:intro}

Let $A$ be a commutative ring and $k\in\mathbb{Z}^+$.
Fix a \textbf{shape} $\vec{s}=(n_1,\ldots,n_k)\in(\mathbb{Z}^+)^k$ and set $n=\min(\vec{s})-1$.
We construct the diagonal simplicial tensor module $X_\bullet(\vec{s};A)$ (DSTM), a simplicial $A$-module
whose $p$-simplices are functions on cosimplicial index sets $I_p(\vec{s})\subseteq\mathbb{N}^k$.

This construction generalizes Quillen's diagonal on a double simplicial group~\cite{Quillen1966}
to $k$ commuting simplicial directions in the category $\mathbf{Mod}_A$ and is compatible with
Dold--Kan normalization in the sense of Dold, Kan, May, Weibel, and Goerss--Jardine~\cite{Dold1958,Kan1958,MaySimplicial,Weibel,GoerssJardine}.


In particular, the diagonal simplicial module $X_\bullet(\vec{s};A)$ is obtained from
$X_\bullet(\vec{s};\mathbb{Z})$ by base change, and the normalized Moore complexes satisfy
\[
N_\bullet\bigl(X_\bullet(\vec{s};A)\bigr)
  \cong N_\bullet\bigl(X_\bullet(\vec{s};\mathbb{Z})\bigr)\otimes_{\mathbb{Z}} A
\]
for all commutative rings $A$.



We analyze the horn kernels
\[
R_{p,j}(X)=\bigcap_{i\neq j}\ker\bigl(d_i:X_p\to X_{p-1}\bigr),
\]
and show that their combinatorics controls the homotopy-theoretic behavior of the DSTM.
For the simplicial dimension $n=\min(\vec{s})-1$ we prove that $X_\bullet(\vec{s};A)$ is
a strict algebraic $n$-hypergroupoid in this sense if and only if the tensor order satisfies $k=n$.
The transition from non-unique to unique
horn fillers occurs exactly at $p=n$, and this threshold is determined by the tensor order.



We briefly outline the main aspects of the paper: combinatorics and classification of the DSTM, 
and the algebraic geometry of generated submodules.


\subsection*{Combinatorics and Classification}


In the abelian setting of simplicial modules, horn fillers always exist (the Kan condition is automatic).
The obstruction to the uniqueness of fillers is captured by the horn kernel $R_{p,j}$.
Following terminology from Duskin--Glenn~\cite{Duskin1979,Glenn1982} and the $n$Lab,
a simplicial module is a \textbf{strict algebraic $n$-hypergroupoid} if fillers are unique
in dimensions strictly greater than $n$ ($R_{p,j}=0$ for $p>n$), and non-unique in dimension $n$ ($R_{n,j}\ne 0$).


We characterize $R_{n,j}$ combinatorially via a basis of ``missing indices.''
The Horn Non-Degeneracy Lemma proves $R_{n,j}\cap D_n=\{0\}$, where $D_n$ denotes the degenerate submodule in degree $n$,
yielding the decomposition $X_n = R_{n,j} \oplus D_n$.
Counting missing indices gives exact rank formulas for $R_{p,j}$ via inclusion--exclusion on the product sets
$I_p = \prod_a [M_a(p)]$.
In the constant shape case $n_a=n+1$ for all $a$, so $M_a(p)=p$ and $I_p=[p]^k$, and these rank formulas
specialize to finite differences of $x^k$ and can be expressed using Stirling numbers of the second kind
$\StirlingII{k}{m}$~\cite{StanleyEC1}.
By analyzing when the horn kernel vanishes, we obtain our main classification theorem:
$X_\bullet(\vec{s};A)$ is a strict algebraic $n$-hypergroupoid if and only if $k=n$.


\subsection*{Algebraic Geometry of Generated Submodules}
When $A$ is an infinite field $K$, we work throughout with submodules generated by a single tensor $T\in X_n(\vec{s};K)$;
we do not attempt to parametrize arbitrary subcomplexes. Such a generated submodule $\langle T \rangle$ is
encoded by its kernel sequence
\[
K(T)_\bullet := (\ker f_{T,p})_p \subseteq K[\Delta^n]_\bullet,
\]
where $f_T$ is the realization map from Section~\ref{sec:generated}.

We define a moduli map $\Psi$ from a Zariski-open locus $\mathcal{U} \subset X_n(\vec{s};K)$
to a product of Grassmannians by sending $T$ to the collection of subspaces $K(T)_p \subseteq K[\Delta^n]_p$.
The image $\mathcal{M}(\vec{s})$ lies within a closed incidence subvariety defined by simplicial compatibility
conditions, which we show are linear in the Segre--Pl\"ucker coordinates.
This reduces the classification of these generated submodules to the geometry of these incidence conditions
and of the index collision maps $\mathcal{I}_p$.




\subsection*{Organization}

Section~\ref{sec:dstm} establishes notation and the DSTM construction.
Section~\ref{sec:horns} characterizes missing indices and the horn kernel basis.
Section~\ref{sec:normalization} develops the Moore filler algorithm and normalization.
Section~\ref{sec:combinatorics} establishes rank formulas and the classification theorem.
Section~\ref{sec:hornnondegeneracylemma} proves the Horn Non-Degeneracy Lemma and the
decomposition $X_n = R_{n,j} \oplus D_n$.
Section~\ref{sec:dichotomy} develops the homology dichotomy and the hypergroupoid classification.
Section~\ref{sec:generated} develops the algebraic geometry of generated submodules.
\ref{app:filtration} proves contractibility via an explicit equivariant homotopy.


