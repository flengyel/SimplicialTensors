\section{The Geometry of Generated Subobjects and Moduli Spaces}
\label{sec:generated}

We analyze the isomorphism classes of the simplicial submodules
$\langle T \rangle \subseteq X_\bullet(\vec s; K)$ generated by a tensor
$T\in X_n$. We assume the ground ring $K$ is an infinite field.


\subsection{Realization Maps and Kernel Sequences}

Let $C_\bullet := K[\Delta^n]$ be the standard simplicial $K$-module representing the
$n$-simplex. In degree $p$, $C_p = K[\Delta^n]_p$ is the free $K$-vector space
with basis the set of morphisms $\Delta([p],[n])$. The dimension is
$S_p := \dim(C_p) = \binom{n+p+1}{p+1}$. Let
$\iota_n \in C_n$ be the generator corresponding to the identity map.

\begin{definition}[Realization Map and Kernel Sequence]
For $T\in X_n(\vec{s};K)$, the \textbf{realization map}
$f_T: C_\bullet \to X_\bullet(\vec s; K)$ is the unique morphism of
simplicial modules sending $\iota_n$ to $T$.
The \textbf{kernel sequence} of $T$ is the simplicial submodule
\[
K(T)_\bullet := \ker(f_T) \subset C_\bullet,
\]
defined degreewise by $K(T)_p := \ker(f_{T,p}) \subseteq C_p$.
\end{definition}

The image is $\langle T \rangle_\bullet = \operatorname{im}(f_T)$, and the
realization map induces an isomorphism of simplicial $K$-modules
\[
C_\bullet / K(T)_\bullet \;\xrightarrow{\ \cong\ }\; \langle T \rangle_\bullet.
\]
The isomorphism class of $\langle T \rangle_\bullet$ is therefore determined by
its kernel sequence $K(T)_\bullet$.

\begin{remark}[Finite Generation]\label{rem:finite_generation}
Since $K[\Delta^n]$ is generated in degree $n$, any simplicial submodule is
determined by its components up to degree $n$.
\end{remark}


\begin{proposition}[Quotients and generated submodules]\label{prop:isomorphism-classes-and-orbits}
For each $T\in X_n(\vec{s};K)$, the realization map
$f_T:C_\bullet\to X_\bullet(\vec s;K)$ induces an isomorphism
\[
C_\bullet / K(T)_\bullet \xrightarrow{\ \cong\ } \langle T\rangle_\bullet.
\]
In particular, all invariants of the generated submodule $\langle T\rangle_\bullet$
can be computed from its kernel sequence $K(T)_\bullet \subseteq C_\bullet$.
\end{proposition}

\begin{proof}
Since $K(T)_\bullet = \ker(f_T)$, the universal property of quotients gives a
simplicial map
\[
C_\bullet/K(T)_\bullet \longrightarrow \langle T\rangle_\bullet
\]
which is an isomorphism in each degree, because $K(T)_p = \ker(f_{T,p})$ and
$\langle T\rangle_p = \operatorname{im}(f_{T,p})$.
\end{proof}




\begin{remark}
For our purposes it suffices that each generated submodule $\langle T\rangle_\bullet$
arises as a quotient $C_\bullet/K(T)_\bullet$, and that all invariants we study
(e.g.\ homology groups and the incidence conditions defining the moduli space)
depend only on the kernel sequence $K(T)_\bullet \subseteq C_\bullet$.
We make no assertion that isomorphic quotients have isomorphic kernels, nor that
every isomorphism between quotients lifts to an automorphism of $C_\bullet$.
\end{remark}



\subsection{Index Collisions and the Realization Matrix}

Let $R_p := \dim(X_p)$.

\begin{definition}[Index collision map]\label{def:index_collision_map}
For each $p\ge 0$, the \textbf{index collision map} is
\[
\mathcal{I}_p: I_p \times \Delta([p],[n]) \longrightarrow I_n,
\qquad
\mathcal{I}_p(m, \alpha) := I_\bullet(\alpha)(m).
\]
\end{definition}

\begin{definition}[Realization matrix]
The \textbf{realization matrix} $M_{T,p} \in \operatorname{Mat}_{R_p\times S_p}(K)$
represents $f_{T,p}$. Its entry in row $m \in I_p$ and column
$\alpha \in \Delta([p],[n])$ is
\[
M_{T,p}[m, \alpha]
= (f_{T,p}(\alpha))(m)
= T_{\mathcal{I}_p(m, \alpha)}.
\]
\end{definition}

A \textbf{collision} in degree $p$ is a pair
$(m,\alpha)\neq (m',\alpha')$ with
$\mathcal{I}_p(m,\alpha)=\mathcal{I}_p(m',\alpha')$.
Collisions force the corresponding entries of $M_{T,p}$ to agree for every
$T$, and are the source of rank loss in the symbolic matrices
$M_{T_{\mathrm{sym}},p}$.


\subsection{The Moduli Map and Grassmannians}

Let $\mathcal{V} = \{v_u\}_{u \in I_n}$ be indeterminates and
$\mathbb{K} := K(\mathcal{V})$ the function field. The symbolic tensor
$T_{\text{sym}}$ has entries $v_u$.

\begin{definition}[Generic Rank and Kernel Dimension]
The \textbf{generic rank} $R'_p(\vec s)$ is the rank of the symbolic
matrix $M_{T_{\text{sym}},p}$ over $\mathbb{K}$.
The \textbf{generic kernel dimension} is
$K'_p(\vec s) := S_p - R'_p(\vec s)$.
\end{definition}

\begin{definition}[Moduli Map]
The \textbf{generic locus} $\mathcal{U} \subset X_n(\vec s;K)$ is the
non-empty Zariski open set where $\operatorname{rank}(M_{T,p}) = R'_p(\vec s)$
for all $p$.

The \textbf{Grassmannian associated to the shape} $\vec s$ is
\[
\operatorname{Gr}(\vec s)
:= \prod_{\substack{0\le p\le n\\ K'_p(\vec s)>0}}
   \operatorname{Gr}\bigl(K'_p(\vec{s}), K[\Delta^n]_p\bigr).
\]
By Remark~\ref{rem:finite_generation}, a simplicial submodule of
$C_\bullet=K[\Delta^n]$ is determined by its components in degrees
$0\le p\le n$, so recording the kernels $K(T)_p=\ker f_{T,p}$ in this
range determines the entire kernel sequence $K(T)_\bullet$.

The \textbf{Moduli Map} $\Psi$ is the algebraic map:
\[
\Psi: \mathcal{U} \longrightarrow \operatorname{Gr}(\vec s),
\quad
T \mapsto (\ker f_{T,p})_{0\le p\le n}.
\]

The \textbf{Moduli Space of Kernel Sequences} $\mathcal{M}(\vec s)$ is
the image $\Psi(\mathcal{U})$.
\end{definition}


\subsection{Injectivity Analysis and Examples}

\begin{lemma}\label{lem:I_bullet_preserves_mono}
The functor $I_\bullet: \Delta \to \mathbf{Set}$ preserves monomorphisms.
\end{lemma}
\begin{proof}
Monomorphisms in $\Delta$ are generated by coface maps $\delta_i$.
$I_\bullet(\delta_i^p) = \Delta_i^p$ is a product of injections, hence
injective.
\end{proof}

\begin{theorem}[Injectivity and Dominance at $p=0$]\label{thm:index_collision_injectivity}
In the DSTM:
\begin{enumerate}
    \item The map $\mathcal{I}_0$ is injective. $R'_0 = \min(R_0, S_0)$.
    \item If $K'_0>0$, the projection
          $\Psi_0: \mathcal{U} \to \operatorname{Gr}(K'_0, S_0)$ is dominant.
\end{enumerate}
\end{theorem}
\begin{proof}
(1) $S_0=n+1$. Let $a_0$ be an axis where $n_{a_0}=n+1$. Then $M_{a_0}(0)=0$.
For $m \in I_0$, $m_{a_0}=0$. Let $\alpha_i:[0]\to[n]$ be the vertex map
$\alpha_i(0)=i$. The $a_0$-th coordinate of $\mathcal{I}_0(m, \alpha_i)$ is
$\alpha_i(m_{a_0}) = i$. If
$\mathcal{I}_0(m, \alpha_i) = \mathcal{I}_0(m', \alpha_{i'})$, then
$i=i'$. By Lemma~\ref{lem:I_bullet_preserves_mono}, $I_\bullet(\alpha_i)$
is injective, so $m=m'$.

(2) Since $\mathcal{I}_0$ is injective, the entries of $M_{T_{\text{sym}},0}$
are distinct variables. The map
$L_0: X_n \to \operatorname{Mat}_{R_0\times S_0}(K)$ sending $T$ to
$M_{T,0}$ is a projection onto these coordinates, hence surjective. The
map from the space of matrices of maximal rank $R'_0$ to the Grassmannian
of their kernels is dominant. $\Psi_0$ is the composition of $L_0$
(restricted to $\mathcal{U}$) and this dominant map.
\end{proof}

\begin{lemma}[Non-injectivity for $p\ge1$]\label{lem:noninj-all-p>0}
For $p\ge 1$, $\mathcal{I}_p$ is not injective.
\end{lemma}
\begin{proof}
Let $\alpha_0:[p]\to[n]$ be the constant map to $0$. It factors as
$\alpha_0 = \iota_0 \circ \pi_0$. Since $p\ge 1$, $|I_p| > |I_0|$.
The map $I_\bullet(\pi_0): I_p \to I_0$ cannot be injective. There exist
$m \ne m'$ such that $I_\bullet(\pi_0)(m) = I_\bullet(\pi_0)(m')$.
Applying $I_\bullet(\iota_0)$ yields
$\mathcal{I}_p(m, \alpha_0) = \mathcal{I}_p(m', \alpha_0)$.
\end{proof}

\begin{example}[Rank Drop: Shape (2,2)]\label{ex:rank_drop_22}
$n=1, k=2$. $S=(2, 3)$. $R=(1, 4)$.
$p=1$. $S_1=3, R_1=4$. The morphisms $\Delta([1],[1])$ are
$\{\mathrm{id}, (0,0), (1,1)\}$. $I_1=[1]^2$. The symbolic realization
matrix $M_{T_{\text{sym}},1}$ is:
\[
M = \begin{pmatrix}
v_{00} & v_{00} & v_{11} \\
v_{01} & v_{00} & v_{11} \\
v_{10} & v_{00} & v_{11} \\
v_{11} & v_{00} & v_{11}
\end{pmatrix}.
\]
All four $3\times 3$ minors vanish identically over $\mathbb{K}$. The
generic rank is $R'_1=2$. $K'_1=1$.
\end{example}

\subsection{The Incidence Variety and Segre--Plücker Embedding}

The image $\mathcal{M}(\vec s)$ lies within the closed incidence
subvariety $\mathbf{Gr}^{\mathrm{simp}}(\vec s) \subset \operatorname{Gr}(\vec s)$
defined by the simplicial compatibility conditions.

\begin{theorem}[Global Structure of the Moduli Space]\label{thm:incidence-global}
$\mathcal{M}(\vec s)$ is a constructible set contained in
$\mathbf{Gr}^{\mathrm{simp}}(\vec s)$.
The projection
$\mathcal M(\vec s)\to \operatorname{Gr}(K'_0(\vec s),S_0)$ is dominant
(if $K'_0(\vec s)>0$).
The moduli space $\mathcal{M}(\vec s)$ is irreducible and unirational.
\end{theorem}
\begin{proof}
$\mathcal{U}$ is an open subset of an affine space, hence irreducible.
$\mathcal{M}(\vec{s})$ is the image of an irreducible variety under a
rational map, hence it is constructible (Chevalley's theorem),
irreducible, and unirational. Dominance at $p=0$ follows from
Theorem~\ref{thm:index_collision_injectivity}.
\end{proof}

\begin{proposition}[Segre--Plücker linearity]\label{prop:segre-linear}
Embed $\operatorname{Gr}(\vec s)$ into a projective space via the
composition of the Plücker embeddings
$\iota_p : \operatorname{Gr}(K'_p,S_p) \hookrightarrow
\mathbb{P}(\Lambda^{K'_p} K[\Delta^n]_p)$ and the Segre embedding
$\Sigma$. The image $\Sigma(\mathbf{Gr}^{\mathrm{simp}}(\vec s))$ is the
intersection of $\Sigma(\operatorname{Gr}(\vec s))$ with a linear
subspace of the ambient projective space.
\end{proposition}

\begin{proof}
We analyze the condition $d_i(K_p) \subseteq K_{p-1}$. Let
$C_q = K[\Delta^n]_q$. Let $\omega_p$ and $\omega_{p-1}$ be the Plücker
coordinates representing $K_p$ and $K_{p-1}$.
The condition is equivalent to
$\Lambda^{K'_p} d_i(\omega_p) \wedge \eta_{p-1} = 0$, where $\eta_{p-1}$
is a volume element of a complement to $K_{p-1}$. Under the Segre
embedding, these bilinear equations in the Plücker coordinates become
linear equations in the homogeneous coordinates of the tensor product
space.
\end{proof}

\begin{proposition}[Determinantal structure of the incidence variety]\label{prop:determinantal}
Let $C_p := K[\Delta^n]_p$. On $\operatorname{Gr}(\vec s)$, let
$\mathcal{S}_p \subset C_p \otimes \mathcal{O}_{\operatorname{Gr}(\vec s)}$
denote the pullback of the tautological rank-$K'_p(\vec s)$ subbundle on
$\operatorname{Gr}(K'_p(\vec s),C_p)$ along the projection
$\operatorname{Gr}(\vec s) \to \operatorname{Gr}(K'_p(\vec s),C_p)$, and
let $\mathcal{S}_{p-1}$ be defined similarly in degree $p-1$.

For each pair $(p,i)$ with $K'_p(\vec s)>0$, the face map
$d_i : C_p \to C_{p-1}$ induces a morphism of vector bundles
\[
\varphi_{p,i} : \mathcal{S}_p \longrightarrow
 \bigl(C_{p-1} \otimes \mathcal{O}_{\operatorname{Gr}(\vec s)}\bigr)
/\,\mathcal{S}_{p-1}.
\]
Then the incidence variety $\mathbf{Gr}^{\mathrm{simp}}(\vec s)$ is the
scheme-theoretic intersection
\[
\mathbf{Gr}^{\mathrm{simp}}(\vec s)
 = \bigcap_{p,i} \{\,\varphi_{p,i} = 0\,\},
\]
i.e.\ it is the intersection of the rank-$0$ determinantal loci of the
bundle maps $\varphi_{p,i}$. In particular, $\mathbf{Gr}^{\mathrm{simp}}(\vec s)$
is a determinantal subvariety of $\operatorname{Gr}(\vec s)$.
\end{proposition}

\begin{proof}
A point of $\operatorname{Gr}(\vec s)$ is a tuple of subspaces
$(K_p)_p$ with $K_p \subset C_p$ and $\dim K_p = K'_p(\vec s)$ whenever
$K'_p(\vec s)>0$. By construction of the tautological subbundle,
the fiber $(\mathcal{S}_p)_x$ over a point $x$ corresponding to
$(K_p)_p$ is exactly $K_p$, and similarly $(\mathcal{S}_{p-1})_x = K_{p-1}$.

The fixed linear map $d_i : C_p \to C_{p-1}$ induces a morphism of
vector bundles
\[
d_i \otimes \mathrm{id} : C_p \otimes \mathcal{O}_{\operatorname{Gr}(\vec s)}
 \longrightarrow C_{p-1} \otimes \mathcal{O}_{\operatorname{Gr}(\vec s)}.
\]
Composing with the quotient map
$C_{p-1} \otimes \mathcal{O}_{\operatorname{Gr}(\vec s)}
 \twoheadrightarrow (C_{p-1} \otimes \mathcal{O}_{\operatorname{Gr}(\vec s)})/\mathcal{S}_{p-1}$
and restricting to $\mathcal{S}_p \subset C_p \otimes \mathcal{O}$ gives
a bundle map
\[
\varphi_{p,i} : \mathcal{S}_p \to
 (C_{p-1} \otimes \mathcal{O}_{\operatorname{Gr}(\vec s)})/\mathcal{S}_{p-1}.
\]

At a point $x=(K_p)_p$, the fiber map
$(\varphi_{p,i})_x : K_p \to C_{p-1}/K_{p-1}$ is the composition
\[
K_p \hookrightarrow C_p \xrightarrow{d_i} C_{p-1} \twoheadrightarrow C_{p-1}/K_{p-1}.
\]
Thus $(\varphi_{p,i})_x = 0$ if and only if $d_i(K_p) \subseteq K_{p-1}$.
By definition, $\mathbf{Gr}^{\mathrm{simp}}(\vec s)$ is the locus of tuples satisfying
these inclusions for all $(p,i)$.

In local trivializations, $\varphi_{p,i}$ is represented by a matrix of regular functions,
and the condition $(\varphi_{p,i})_x=0$ corresponds to the vanishing of all entries
(all $1\times 1$ minors).
Globally, this condition defines the zero locus of the section $\varphi_{p,i}$ of the bundle
$\mathcal{H}om(\mathcal{S}_p, \mathcal{Q}_{p-1})$, where we define the quotient bundle
$\mathcal{Q}_{p-1} := (C_{p-1} \otimes \mathcal{O}_{\operatorname{Gr}(\vec s)})/\mathcal{S}_{p-1}$.
In terms of homogeneous coordinates, this condition is bilinear in the Plücker coordinates
of the factors (and thus linear in the Segre coordinates), confirming that
$\mathbf{Gr}^{\mathrm{simp}}(\vec s)$ is a determinantal subvariety.
\end{proof}


\subsection{Homology of Generated Subobjects}

Fix $T\in X_n(\vec s;K)$ and write $C_\bullet := K[\Delta^n]$ and
$K_\bullet := K(T)_\bullet = \ker(f_T) \subseteq C_\bullet$.
We analyze $H_\bullet(\langle T \rangle_\bullet)$ using the short exact sequence
\[
0 \longrightarrow K_\bullet \longrightarrow C_\bullet \longrightarrow
\langle T \rangle_\bullet \longrightarrow 0.
\]
Recall that $H_p(C_\bullet)=0$ for $p>0$ and $H_0(C_\bullet)\cong K$.

\begin{proposition}[Homology of the generated subobject]\label{prop:homology-strata}
Let $B_p(C_\bullet)$ and $B_p(K_\bullet)$ denote the boundary subspaces in
$C_\bullet$ and $K_\bullet$, respectively. Then:
\begin{enumerate}
    \item For $p\ge 2$, there is a natural isomorphism
          \[
          H_p(\langle T \rangle_\bullet) \cong H_{p-1}(K_\bullet).
          \]
    \item $H_1(\langle T \rangle_\bullet) \cong
          \bigl(K_0 \cap B_0(C_\bullet)\bigr) / B_0(K_\bullet)$.
    \item $H_0(\langle T \rangle_\bullet) \cong
          C_0 / \bigl(K_0 + B_0(C_\bullet)\bigr)$.
\end{enumerate}
\end{proposition}

\begin{proof}
Applying homology to
\[
0 \to K_\bullet \to C_\bullet \to \langle T\rangle_\bullet \to 0
\]
gives a long exact sequence. For $p\ge 2$ we have
$H_p(C_\bullet)=H_{p-1}(C_\bullet)=0$, so the connecting homomorphism
induces an isomorphism $H_p(\langle T\rangle_\bullet)\cong H_{p-1}(K_\bullet)$,
which proves (1).

For $p=1$ and $p=0$, the long exact sequence reads
\[
0 \to H_1(\langle T\rangle_\bullet) \to H_0(K_\bullet) \to H_0(C_\bullet)
\to H_0(\langle T\rangle_\bullet) \to 0,
\]
and the standard identification of $H_0$ as cycles modulo boundaries in degree
$0$ yields the descriptions in (2) and (3).
\end{proof}

\begin{corollary}[Generic connectivity]\label{cor:generic_connectivity}
If $K'_0(\vec s)>0$, then for a generic tensor $T \in \mathcal{U}$,
$H_0(\langle T \rangle_\bullet)=0$.
\end{corollary}

\begin{proof}
$B_0(C_\bullet)$ is a hyperplane in $C_0$. By
Proposition~\ref{prop:homology-strata}, we have
$H_0(\langle T \rangle_\bullet)\ne 0$ if and only if
$K_0(T) \subseteq B_0(C_\bullet)$. This condition cuts out a proper closed
subvariety of $\operatorname{Gr}(K'_0(\vec s),S_0)$. Since $\Psi_0$ is
dominant (Theorem~\ref{thm:index_collision_injectivity}), the generic
kernel $K_0(T)$ avoids this locus.
\end{proof}


\subsection{\texorpdfstring{Example: Shape $(3,3)$}{Example: Shape (3,3)}}
\label{ex:shape_33}

Let $\vec{s}=(3,3)$, so $k=2$ and $n=2$. Then $X_2$ consists of $3\times 3$
matrices. We write $T=(v_{ij})_{0\le i,j\le 2}$. The dimensions are
\[
S=(3,6,10),\qquad R=(1,4,9).
\]

We fix the following bases:
\begin{itemize}
\item For $X_2$, the basis $I_2 = [2]^2$ in lexicographic order:
\[
(0,0),(0,1),(0,2),(1,0),(1,1),(1,2),(2,0),(2,1),(2,2).
\]
\item For $K[\Delta^2]_1$, the basis of monotone maps
      $[1]\to[2]$ ordered as
\[
(0,0),\ (0,1),\ (0,2),\ (1,1),\ (1,2),\ (2,2).
\]
\item For $K[\Delta^2]_2$, the basis of monotone maps $[2]\to[2]$
      ordered as the ten non-decreasing triples
      $(a_0,a_1,a_2)$ with $0\le a_0\le a_1\le a_2\le 2$:
\[
(0,0,0), (0,0,1), (0,0,2), (0,1,1), (0,1,2),
(0,2,2), (1,1,1), (1,1,2), (1,2,2), (2,2,2).
\]
\end{itemize}

\paragraph{Degree $p=0$.}
Here $R_0=1$ and $S_0=3$. The realization matrix is
\[
M_{T,0} = \begin{pmatrix} v_{00} & v_{11} & v_{22} \end{pmatrix}.
\]
For generic $T$, its rank is $1$, so $K'_0 = 2$.
The kernel is the $2$-dimensional subspace of $K^3$ of solutions to
\[
v_{00}x_0 + v_{11}x_1 + v_{22}x_2 = 0.
\]
On the open chart $v_{00}\neq 0$, it is spanned by
\[
(-v_{11}/v_{00},\, 1,\,0),\qquad
(-v_{22}/v_{00},\, 0,\,1).
\]
Thus $K_0(T)$ depends only on the projective class
$[v_{00}:v_{11}:v_{22}] \in \mathbb{P}^2$.

\paragraph{Degree $p=1$.}
Here $R_1=4$ and $S_1=6$. With the bases above, the realization matrix is
\[
M_{T,1} =
\begin{pmatrix}
v_{00} & v_{00} & v_{00} & v_{11} & v_{11} & v_{22} \\
v_{00} & v_{01} & v_{02} & v_{11} & v_{12} & v_{22} \\
v_{00} & v_{10} & v_{20} & v_{11} & v_{21} & v_{22} \\
v_{00} & v_{11} & v_{22} & v_{11} & v_{22} & v_{22}
\end{pmatrix}.
\]

Two obvious kernel vectors can be written purely in terms of the diagonal
entries. Set
\[
c^{(1)}_1 := \bigl(-v_{11}/v_{00},\, 0,\,0,\, 1,\, 0,\, 0\bigr)^T,
\qquad
c^{(1)}_2 := \bigl(-v_{22}/v_{00},\, 0,\,0,\, 0,\, 0,\, 1\bigr)^T.
\]
A direct calculation shows
\[
M_{T,1}\,c^{(1)}_1 = 0,
\qquad
M_{T,1}\,c^{(1)}_2 = 0
\]
as identities in the polynomial ring $K[v_{ij}]$ (on the open set
$v_{00}\neq 0$). Hence $K'_1 \ge 2$.

On the other hand, taking e.g.
\[
T =
\begin{pmatrix}
1 & 1 & 1 \\
2 & 2 & 1 \\
2 & 2 & 3
\end{pmatrix},
\]
one obtains
\[
M_{T,1} =
\begin{pmatrix}
1 & 1 & 1 & 2 & 2 & 3 \\
1 & 1 & 1 & 2 & 1 & 3 \\
1 & 2 & 2 & 2 & 2 & 3 \\
1 & 2 & 3 & 2 & 3 & 3
\end{pmatrix},
\]
which has rank $4$ (one checks that some $4\times 4$ minor is non-zero).
Thus the generic rank is $R'_1=4$, and therefore $K'_1 = S_1 - R'_1 = 2$.

In particular, for generic $T$, the kernel $K_1(T) = \ker f_{T,1}$ is
spanned by $c^{(1)}_1$ and $c^{(1)}_2$, so it depends only on
$(v_{00},v_{11},v_{22})$.

\paragraph{Degree $p=2$.}
Here $R_2=9$ and $S_2=10$. With the bases chosen above, the symbolic
realization matrix $M_{T,2}$ is
\[
M_{T,2} =
\begin{pmatrix}
v_{00} & v_{00} & v_{00} & v_{11} & v_{11} & v_{22} & v_{11} & v_{11} & v_{22} & v_{22} \\
v_{00} & v_{00} & v_{00} & v_{01} & v_{01} & v_{02} & v_{11} & v_{11} & v_{12} & v_{22} \\
v_{00} & v_{00} & v_{00} & v_{10} & v_{10} & v_{20} & v_{11} & v_{11} & v_{21} & v_{22} \\
v_{00} & v_{00} & v_{01} & v_{11} & v_{12} & v_{22} & v_{11} & v_{12} & v_{22} & v_{22} \\
v_{00} & v_{01} & v_{02} & v_{11} & v_{12} & v_{22} & v_{11} & v_{12} & v_{22} & v_{22} \\
v_{00} & v_{02} & v_{02} & v_{12} & v_{12} & v_{22} & v_{12} & v_{12} & v_{22} & v_{22} \\
v_{00} & v_{00} & v_{10} & v_{11} & v_{21} & v_{22} & v_{11} & v_{21} & v_{22} & v_{22} \\
v_{00} & v_{10} & v_{20} & v_{11} & v_{21} & v_{22} & v_{11} & v_{21} & v_{22} & v_{22} \\
v_{00} & v_{20} & v_{20} & v_{21} & v_{21} & v_{22} & v_{21} & v_{21} & v_{22} & v_{22}
\end{pmatrix}.
\]

Again there are two explicit kernel vectors depending only on the
diagonal. Set
\[
c^{(2)}_1 :=
\bigl(-v_{11}/v_{00},\, 0,\,0,\,0,\,0,\,0,\, 1,\,0,\,0,\,0\bigr)^T,
\qquad
c^{(2)}_2 :=
\bigl(-v_{22}/v_{00},\, 0,\,0,\,0,\,0,\,0,\, 0,\,0,\,0,\,1\bigr)^T.
\]
A direct substitution into the matrix above shows
\[
M_{T,2}\,c^{(2)}_1 = 0,
\qquad
M_{T,2}\,c^{(2)}_2 = 0
\]
as polynomial identities (again on the open chart $v_{00}\neq 0$).
Thus $K'_2 \ge 2$ and $R'_2 \le 10-2 = 8$.

To see that the generic rank is exactly $8$, we exhibit a single tensor
for which $M_{T,2}$ has rank $8$. Take for example
\[
T =
\begin{pmatrix}
1 & 1 & 1 \\
2 & 2 & 1 \\
2 & 2 & 3
\end{pmatrix}.
\]
Substituting these values into $M_{T,2}$ gives a $9\times 10$ numerical
matrix with rank $8$. Since rank is upper semicontinuous in families,
this shows that the generic rank is $R'_2=8$, and therefore
$K'_2 = S_2 - R'_2 = 2$.

Moreover, the two kernel generators $c^{(2)}_1,c^{(2)}_2$ depend only on
$v_{00},v_{11},v_{22}$, not on the off-diagonal entries.

\medskip

Collecting the three degrees, we have:

\begin{proposition}[Generic kernel dimensions for shape $(3,3)$]
For $\vec{s}=(3,3)$, the generic kernel dimensions are
\[
K'(\vec{s}) = (K'_0,K'_1,K'_2) = (2,2,2).
\]
On the open chart $v_{00}\neq 0$, the kernels $K_p(T)$ are generated by
vectors whose coordinates are rational functions of $v_{11}/v_{00}$ and
$v_{22}/v_{00}$ only.
\end{proposition}

\begin{corollary}[Moduli space for shape $(3,3)$]
For $\vec{s}=(3,3)$, the moduli space $\mathcal{M}(\vec{s})$ is a surface
of dimension $2$, birational to $\mathbb{P}^2$.
\end{corollary}

\begin{proof}
The generic kernel vector in degree $0$ is the hyperplane
\[
K_0(T) = \{(x_0,x_1,x_2)\mid v_{00}x_0+v_{11}x_1+v_{22}x_2=0\}
\subset K^3.
\]
Thus the point $K_0(T) \in \operatorname{Gr}(2,3)$ determines the
projective class $[v_{00}:v_{11}:v_{22}] \in \mathbb{P}^2$, and hence
determines the ratios $v_{11}/v_{00}$ and $v_{22}/v_{00}$ on the chart
$v_{00}\neq 0$. By the explicit formulas above, $K_1(T)$ and $K_2(T)$
are then uniquely determined. Therefore, on this chart, the moduli map
$\Psi$ factors through the projection
\[
X_2 \dashrightarrow \mathbb{P}^2,\qquad
T \longmapsto [v_{00}:v_{11}:v_{22}],
\]
and the induced map $\mathbb{P}^2 \dashrightarrow \mathcal{M}(\vec{s})$
is birational onto its image. Hence $\mathcal{M}(\vec{s})$ has dimension
$2$ and is birational to $\mathbb{P}^2$.
\end{proof} 