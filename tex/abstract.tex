\begin{abstract}
Let $A$ be a commutative ring, $k\in\mathbb{Z}^+$, and $\vec{s}\in(\mathbb{Z}^+)^k$ with
$n = \min(\vec{s}) - 1$.
To $\vec{s}$ we attach a diagonal simplicial tensor module $X_\bullet(\vec{s};A)$: a simplicial
$A$-module whose $p$-simplices are functions on a cosimplicial index set $I_p(\vec{s})\subseteq\mathbb{N}^k$.
This extends Quillen’s diagonal for double simplicial modules to $k$ commuting directions and is
compatible with Dold–Kan normalization.
For $k=2$ and $s_1=s_2$, the normalized chain complex recovers the usual combinatorial Laplacian
of a finite graph together with effective resistance and Foster-type identities.

We analyze the horn kernels $R_{p,j}(X)$ via “missing indices’’ and show that $R_{p,j}(X)\neq 0$
if and only if $k\ge p$.
In particular, $X_\bullet(\vec{s};A)$ is a strict algebraic $n$-hypergroupoid (in the sense of
Duskin–Glenn) if and only if $k=n$, and a Horn Non-Degeneracy Lemma yields a decomposition
$X_n = R_{n,j}(X)\oplus D_n$.
An explicit shift-and-truncate chain homotopy shows that $X_\bullet(\vec{s};A)$ is contractible;
it is $\operatorname{Stab}(\vec{s})$-equivariant and preserves a natural filtration, so the
associated spectral sequence collapses at $E_1$.

Over an infinite field we classify simplicial submodules generated by a single tensor via kernel
sequences and a moduli map to a product of Grassmannians, obtaining an irreducible and unirational
incidence variety.
Over a subfield of $\mathbb{R}$ we develop combinatorial Hodge theory on $X_\bullet(\vec{s};A)$:
ambient Laplacians are invertible and define effective resistance metrics, and for generated
submodules projected Laplacians encode Betti numbers, Hodge spectra, resistance pseudometrics,
and a DSTM Foster-type trace identity.
\end{abstract}
