\section{Horns, Kernels, and Missing Indices}
\label{sec:horns}

We investigate the structure of face kernels and the uniqueness of fillers for horns in $X_\bullet(\vec{s};A)$.

\subsection{Horns and the Horn Kernel}

Let $p\ge 1$ and $0\le j\le p$. We denote $F_j = [p] \setminus \{j\}$.

\begin{definition}[$(p,j)$-Horn and Filler]
A \textbf{$(p,j)$-horn} in $X_\bullet$ is a tuple $H = (h_i)_{i \in F_j}$ of elements $h_i \in X_{p-1}$ satisfying the \textbf{horn compatibility condition}:
\[
d_i(h_\ell) = d_{\ell-1}(h_i) \quad \text{for all } i < \ell \text{ with } i, \ell \in F_j.
\]
A tensor $T\in X_p$ is a \textbf{filler} of $H$ if $d_i(T) = h_i$ for all $i\in F_j$.
\end{definition}

In the category of simplicial modules, every horn has at least one filler. The uniqueness of fillers is measured by the horn kernel.

\begin{definition}[Horn Kernel]
The \textbf{$(p,j)$-horn kernel} $R_{p,j}$ is the submodule of $X_p$ defined by
\[
R_{p,j} := \bigcap_{i \in F_j} \ker(d_i^p).
\]
\end{definition}

If $T$ and $T'$ are two fillers for the same horn $H$, their difference $T-T' \in R_{p,j}$. The set of fillers $\mathcal{F}(H)$ is an affine space (torsor) modeled on $R_{p,j}$. Fillers are unique if and only if $R_{p,j} = \{0\}$.

\subsection{Structure of the Face Kernels}

We characterize the kernels of the face maps using the standard basis $\{E_m\}_{m\in I_p}$ of $X_p$.

\begin{lemma}[Action on basis elements]\label{lem:face_action}
Let $m \in I_p$. The action of the face map $d_i^p$ on $E_{m}$ is
\[
d_i^p(E_{m}) =
\begin{cases}
E_{m'} & \text{if } m \in \operatorname{im}(\Delta_i^p), \text{ where } m' \text{ is the unique preimage},\\[4pt]
0      & \text{if } m \notin \operatorname{im}(\Delta_i^p).
\end{cases}
\]
\end{lemma}
\begin{proof}
Evaluate at $m'' \in I_{p-1}$: $(d_i^p(E_{m}))(m'') = E_{m}(\Delta_i^p(m'')) = \delta_{m, \Delta_i^p(m'')}$. If $m \notin \operatorname{im}(\Delta_i^p)$, this is zero. If $m \in \operatorname{im}(\Delta_i^p)$, since $\Delta_i^p$ is injective, there is a unique $m'$ such that $\Delta_i^p(m')=m$. Then $d_i^p(E_{m}) = E_{m'}$.
\end{proof}

\begin{lemma}[Injectivity Lemma]\label{lem:injectivity}
If $d_i^p(E_{m_1}) = d_i^p(E_{m_2}) \ne 0$, then $m_1 = m_2$.
\end{lemma}
\begin{proof}
By Lemma~\ref{lem:face_action}, the common value is some $E_{m^*} \in X_{p-1}$. Then $m_1 = \Delta_i^p(m^*)$ and $m_2 = \Delta_i^p(m^*)$.
\end{proof}

\begin{theorem}[Basis of the Face Kernel]\label{thm:kernel_basis}
The kernel of $d_i^p: X_p \to X_{p-1}$ is the free $A$-submodule
\[
\ker(d_i^p) = \operatorname{span}_A\{E_{m} \mid m \in I_p \setminus \operatorname{im}(\Delta_i^p)\}.
\]
\end{theorem}
\begin{proof}
Let $T = \sum_{m \in I_p} a_{m} E_{m} \in X_p$. Applying $d_i^p$ gives
\[
d_i^p(T) = \sum_{m \in \operatorname{im}(\Delta_i^p)} a_{m} d_i^p(E_{m}),
\]
since $d_i^p(E_{m})=0$ if $m \notin \operatorname{im}(\Delta_i^p)$ (Lemma~\ref{lem:face_action}). By the Injectivity Lemma~\ref{lem:injectivity}, the set $\{d_i^p(E_{m}) \mid m \in \operatorname{im}(\Delta_i^p)\}$ consists of distinct standard basis elements in $X_{p-1}$, hence is $A$-linearly independent.

Thus $d_i^p(T)=0$ if and only if $a_{m} = 0$ for all $m \in \operatorname{im}(\Delta_i^p)$.
\end{proof}

\subsection{Missing Indices and the Support Characterization}

\begin{definition}[Missing Index]\label{def:missing_index}
An index $m\in I_p$ is \textbf{missing} from the $(p,j)$-horn if
\[
\forall i \in F_j: m \notin \operatorname{im}(\Delta_i^p).
\]
Let $M_{p,j} \subset I_p$ denote the set of missing indices.
\end{definition}

\begin{theorem}[Support Characterization / Horn Kernel Basis]\label{thm:support_characterization}
The horn kernel $R_{p,j}$ is a free $A$-module with basis $\{E_m\}_{m \in M_{p,j}}$. Consequently, if $T_1$ and $T_2$ are two fillers of the same horn, their difference is supported on the missing indices: $\operatorname{supp}(T_1-T_2) \subseteq M_{p,j}$.
\end{theorem}
\begin{proof}
By Theorem~\ref{thm:kernel_basis}, $\ker(d_i^p)$ is the coordinate subspace spanned by the basis
$B_i = \{E_m \mid m \notin \operatorname{im}(\Delta_i^p)\}$.
The horn kernel $R_{p,j}$ is the intersection of these coordinate subspaces for $i\in F_j$. The intersection of coordinate subspaces is spanned by the intersection of their bases, $\bigcap_{i\in F_j} B_i$.
An element $E_m$ lies in this intersection if and only if $m \notin \operatorname{im}(\Delta_i^p)$ for all $i\in F_j$, which defines $M_{p,j}$.
The consequence regarding the support of the difference follows because $T_1-T_2 \in R_{p,j}$.
\end{proof}

\subsection{Combinatorial Characterization}

We provide a combinatorial interpretation of missing indices. We denote the image of a multi-index $m=(m_1, \dots, m_k)$ by $\operatorname{im}(m) = \{m_1, \dots, m_k\}$.

\begin{proposition}[Characterization of Missing Indices]\label{prop:missing_indices}
An index $m\in I_p$ is missing from the $(p,j)$-horn ($m \in M_{p,j}$) if and only if $\operatorname{im}(m) \supseteq F_j$.
\end{proposition}

\begin{proof}
We first establish the equivalence: $m \in \operatorname{im}(\Delta_i^p) \iff i \notin \operatorname{im}(m)$.
Recall $\Delta_i^p = \prod_{a=1}^k \delta_i^{M_a(p)}$. The image of $\delta_i^q$ is $[q] \setminus \{i\}$.
\begin{align*}
m \in \operatorname{im}(\Delta_i^p) &\iff \forall a: m_a \in \operatorname{im}(\delta_i^{M_a(p)}) \\
&\iff \forall a: m_a \ne i \quad (\text{since } i \in [p] \subseteq [M_a(p)])\\
&\iff i \notin \operatorname{im}(m).
\end{align*}
Taking the contrapositive, $m \notin \operatorname{im}(\Delta_i^p) \iff i \in \operatorname{im}(m)$.

By definition, $m \in M_{p,j}$ if and only if $\forall i \in F_j: m \notin \operatorname{im}(\Delta_i^p)$. Therefore,
\[
m \in M_{p,j} \iff \forall i \in F_j: i \in \operatorname{im}(m)
\iff F_j \subseteq \operatorname{im}(m).
\]
\end{proof}

\begin{corollary}\label{cor:missing_indices_existence}
The horn kernel is non-trivial ($R_{p,j} \ne \{0\}$) if and only if $k \ge p$.
\end{corollary}
\begin{proof}
By Theorem~\ref{thm:support_characterization}, $R_{p,j} \ne \{0\}$ iff $M_{p,j} \ne \emptyset$.

($\Rightarrow$) If $m \in M_{p,j}$, then $k \ge |\operatorname{im}(m)| \ge |F_j| = p$.

($\Leftarrow$) Suppose $k \ge p$. Let $F_j = \{f_1, \dots, f_p\}$. Define $m = (f_1, \dots, f_p, 0, \dots, 0) \in \mathbb{N}^k$. Since $M_a(p)\ge p$, we have $[p]^k \subseteq I_p$, so $m \in I_p$. Since $\operatorname{im}(m) \supseteq F_j$, $m \in M_{p,j}$.
\end{proof}

\subsection{\texorpdfstring{The $\ell$-free Subspace}{The l-free Subspace}}

The combinatorial characterization of $\operatorname{im}(\Delta_\ell^p)$ yields a structural property of the face maps.

\begin{definition}[$\ell$-free subspace]
Define the $\ell$-free subspace
\[
X_p^{(\ell\text{-free})}
:=\operatorname{span}_A\{\,E_{m}:\ \ell \notin \operatorname{im}(m) \,\}\ \subseteq X_p.
\]
\end{definition}

\begin{corollary}[Isomorphism on the $\ell$-free subspace]\label{cor:inj-face}
The face map $d_\ell^p$ induces a linear isomorphism
\[
d_\ell^p:\ X_p^{(\ell\text{-free})}\ \xrightarrow{\ \cong\ }\ X_{p-1}.
\]
\end{corollary}
\begin{proof}
By the proof of Proposition~\ref{prop:missing_indices}, $m \in \operatorname{im}(\Delta_\ell^p) \iff \ell \notin \operatorname{im}(m)$. Thus $X_p^{(\ell\text{-free})}$ is spanned by $\{E_{m} \mid m \in \operatorname{im}(\Delta_\ell^p)\}$. By the Injectivity Lemma~\ref{lem:injectivity}, $d_\ell^p$ is injective on this subspace. Since $\Delta_\ell^p: I_{p-1} \to \operatorname{im}(\Delta_\ell^p)$ is a bijection, Lemma~\ref{lem:face_action} shows that $d_\ell^p$ maps the basis of $X_p^{(\ell\text{-free})}$ bijectively onto the basis of $X_{p-1}$.
\end{proof}

\subsection{Genericity}

Let $T' = \mu_j(\Phi_j(T))$ be the Moore filler of $T$ (see Section~\ref{sec:normalization}). The difference $T-T'$ lies in $R_{p,j}$. By Theorem~\ref{thm:support_characterization}, $\operatorname{supp}(T-T') \subseteq M_{p,j}$.

\begin{definition}[Generic Tensor]
Assume $A$ is an integral domain. A tensor $T\in X_p$ is called \textbf{$(p,j)$-generic} if the difference $T-T'$ has maximal support: $\operatorname{supp}(T-T')=M_{p,j}$. (This requires $k\ge p$).
\end{definition}


\begin{proposition}[Generic Locus]\label{prop:generic_locus}
Assume $A$ is an integral domain and $k\ge p$. The set of $(p,j)$-generic tensors forms a Zariski-open subset $\mathcal{U}_{p,j} \subset X_p$. If moreover $A$ is infinite, then $\mathcal{U}_{p,j}$ is non-empty.
\end{proposition}


\begin{proof}
Let $R(T) = T-T'$. The map $T\mapsto T'$ is $A$-linear (as $\mu_j$ and $\Phi_j$ are linear), hence $R(T)$ is $A$-linear. Write $R(T) = \sum_{m\in M_{p,j}} c_m(T) E_m$. Each coefficient $c_m(T)$ is a linear form in the entries of $T$.

We show that $c_m(T)$ is not identically zero. Consider $T=E_m$ for $m\in M_{p,j}$. Since $E_m \in R_{p,j}$, its horn is $\Phi_j(E_m)=0$. The Moore filler of the zero horn is $T'=\mu_j(0)=0$.
Then $R(E_m) = E_m$, so $c_m(E_m)=1$.

The generic locus is defined by the non-vanishing of these forms:
\[
\mathcal{U}_{p,j}\ :=\ \{\,T\in X_p:\ c_m(T)\neq 0\text{ for all }m\in M_{p,j}\,\}.
\]
The complement is the union of the zero loci of the non-zero forms $c_m$, hence a finite union of proper linear subspaces. Therefore $\mathcal{U}_{p,j}$ is Zariski-open. If $A$ is infinite, a finite union of proper linear subspaces cannot equal $X_p$, so $\mathcal{U}_{p,j}$ is non-empty.
\end{proof} 