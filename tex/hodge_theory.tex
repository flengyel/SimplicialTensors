\section{Combinatorial Hodge Theory and Effective Resistance}
\label{sec:hodge}

We develop Hodge theory on the chain complex $(X_\bullet(\vec{s}; K), \partial)$ of the Diagonal Simplicial Tensor Module. 
The DSTM provides an ambient space for generalized tensor homology. In this section, we work over a subfield $K\subseteq \mathbb{R}$ and equip 
each $X_p$ with the standard inner product making the tensor basis $\{E_m\}_{m \in I_p}$ orthonormal.

\subsection{The Combinatorial Laplacian}

\begin{definition}[Adjoint and Laplacian]
The \textbf{coboundary operator} $\partial_p^*: X_{p-1} \to X_p$ is the adjoint of the boundary $\partial_p$ with respect to the standard inner product.
The \textbf{combinatorial Laplacian} $\Delta_p: X_p \to X_p$ is defined by the Hodge formula:
\[
\Delta_p := \Delta_p^{\mathrm{down}} + \Delta_p^{\mathrm{up}} := \partial_p^*\partial_p + \partial_{p+1}\partial_{p+1}^*.
\]
\end{definition}

\noindent \textbf{Note:} We follow the standard convention that $\Delta_p^{\mathrm{down}} = \partial_p^*\partial_p$ (passing through $X_{p-1}$) and $\Delta_p^{\mathrm{up}} = \partial_{p+1}\partial_{p+1}^*$ (passing through $X_{p+1}$). The operator $\Delta_p$ is self-adjoint and positive semidefinite.

\begin{theorem}[Hodge Decomposition Theorem]
There is an orthogonal direct sum decomposition of the chain space:
\[
X_p = \operatorname{im}(\partial_{p+1}) \oplus \operatorname{im}(\partial_p^*) \oplus \ker(\Delta_p).
\]
The space of \textbf{harmonic forms} $\mathcal{H}_p := \ker(\Delta_p)$ corresponds to the homology:
\[
\mathcal{H}_p = \ker(\partial_p) \cap \ker(\partial_{p+1}^*) \cong H_p(X_\bullet).
\]
\end{theorem}

\begin{remark}
This constitutes a combinatorial Hodge decomposition for the DSTM. It generalizes the graph-theoretic decomposition of Eckmann~\cite{Eckmann1944} and the simplicial decomposition of Horak--Jost~\cite{HorakJost2013}.
\end{remark}

\subsection{Effective Resistance on the Ambient DSTM}

\begin{proposition}[Invertibility of the Laplacian]\label{prop:laplacian-invertible}
For all $p \ge 0$, the Laplacian $\Delta_p$ on the ambient DSTM $X_\bullet(\vec{s}; K)$ is invertible.
\end{proposition}

\begin{proof}
The DSTM is contractible (Theorem~\ref{thm:contractible}), implying $H_p(X_\bullet) = 0$ for all $p$. By the Hodge theorem, $\ker(\Delta_p) \cong H_p(X_\bullet) = \{0\}$. Since $\Delta_p$ is a self-adjoint operator on a finite-dimensional space with trivial kernel, it is positive definite and invertible.
\end{proof}

\begin{definition}[Green's Operator]
For $p \ge 0$, the \textbf{Green's operator} is the inverse Laplacian $G_p := \Delta_p^{-1}$.
\end{definition}

\begin{remark}[Metric Dependence]
While the contracting homotopy $H$ of Theorem~\ref{thm:contractible} satisfies $\partial H + H\partial = \mathrm{id}$,
the Green's operator $G_p$ depends on the metric structure. Generally, $G_p \neq H_{p-1}H_{p-1}^* + H_p^*H_p$; they capture distinct geometric data.
\end{remark}

\begin{definition}[Effective Resistance]
For $p \ge 0$ and any chains $x,y\in X_p$, the \textbf{effective resistance} is
\[
R_p(x, y) := \langle x - y, G_p(x - y) \rangle.
\]
Since $\Delta_p$ is positive definite, this defines a genuine metric (the \textbf{resistance distance}) on $X_p$ given by $d_p(x,y) := \sqrt{R_p(x,y)}$.
\end{definition}

\begin{proposition}[Symmetry Invariance]
The effective resistance $R_p$ and the induced distance $d_p$ are invariant under the action of the stabilizer group $\operatorname{Stab}(\vec{s})$.
\end{proposition}

\begin{proof}
The boundary maps $\partial_p$ are $\operatorname{Stab}(\vec{s})$-equivariant (Lemma~\ref{lem:equivariance-stab}), and the standard inner product is invariant under permutation of axes. Thus, $\Delta_p$ and $G_p$ commute with the group action.
\end{proof}

\subsection{\texorpdfstring{Example: Shape $(2,2)$}{Example: Shape (2,2)}}\label{ex:hodge_22}

We detail the Hodge structure for $\vec{s} = (2,2)$ ($n=1, k=2$). $X_1$ is the space of $2\times 2$ matrices ($R_1=4$). We identify $X_0 \cong K$ (basis $E_{00}^{(0)}$).

\textbf{Step 1: Down Laplacian ($\Delta_1^{\mathrm{down}} = \partial_1^* \partial_1$).}
The boundary $\partial_1 = d_0 - d_1$ acts on the basis as:
$\partial_1 E_{00} = -1, \partial_1 E_{11} = 1$, and $\partial_1 E_{01} = \partial_1 E_{10} = 0$.
The matrix representation is:
\[
[\Delta_1^{\mathrm{down}}] =
\begin{pmatrix}
1 & 0 & 0 & -1 \\
0 & 0 & 0 & 0 \\
0 & 0 & 0 & 0 \\
-1 & 0 & 0 & 1
\end{pmatrix}.
\]

\textbf{Step 2: Up Laplacian ($\Delta_1^{\mathrm{up}} = \partial_2 \partial_2^*$).}
Using the adjoint faces $d_i^*$, we calculate the Gram matrix of the vectors $\partial_2^*(E_m)$ in $X_2$.
For $E_{01}$, the adjoint $\partial_2^*(E_{01})$ is a sum of 3 orthogonal basis vectors (due to distinct images under index maps), so its squared norm is 3.
For $E_{00}$, an \textbf{index collision} occurs: $d_1^*(E_{00}) = d_2^*(E_{00})$. This causes cancellation in the alternating sum $\partial_2^* = d_0^* - d_1^* + d_2^*$, resulting in a single term $\partial_2^*(E_{00}) = E_{11}^{(2)}$, which has squared norm 1.
\[
[\Delta_1^{\mathrm{up}}] =
\begin{pmatrix}
1 & 0 & 0 & 1 \\
0 & 3 & 0 & 0 \\
0 & 0 & 3 & 0 \\
1 & 0 & 0 & 1
\end{pmatrix}.
\]

\textbf{Step 3: Total Laplacian and Green's Operator.}
Summing the matrices yields a diagonal Laplacian:
\[
[\Delta_1] = [\Delta_1^{\mathrm{down}}] + [\Delta_1^{\mathrm{up}}] = \operatorname{diag}(2, 3, 3, 2).
\]
Inverting gives the Green's operator:
\[
[G_1] = \operatorname{diag}\bigl(1/2,\, 1/3,\, 1/3,\, 1/2\bigr).
\]

\textbf{Step 4: Resistance Distances.}
\begin{align*}
R(E_{00}, E_{11}) &= \langle E_{00}-E_{11}, G_1(E_{00}-E_{11}) \rangle = 1/2 + 1/2 = 1. \\
R(E_{01}, E_{10}) &= \langle E_{01}-E_{10}, G_1(E_{01}-E_{10}) \rangle = 1/3 + 1/3 = 2/3.
\end{align*}
\emph{Geometric Interpretation:} The resistance distance between off-diagonal basis elements ($\sqrt{2/3} \approx 0.816$) is strictly smaller than between diagonal elements ($1$). This reflects the higher connectivity of the off-diagonal terms in the simplicial 2-skeleton.

\subsection{Hodge Theory on Generated Submodules}

For a generated submodule $\langle T \rangle$, we restrict the theory to the orthogonal complement of the kernel.

\begin{definition}[Projected Laplacian and Invariants]
Let $C_p := \ker(f_{T,p})^\perp \cong \langle T \rangle_p$. The \textbf{projected Laplacian} is
\[
\Delta_p^C := \partial_{p+1}^C (\partial_{p+1}^C)^* + (\partial_p^C)^* \partial_p^C,
\]
where $\partial^C$ is the compressed boundary map.
We define spectral invariants for the tensor orbit $\langle T \rangle$:
\begin{enumerate}
    \item \textbf{Betti Numbers:} $\beta_p(\langle T \rangle) := \dim \ker(\Delta_p^C)$.
    \item \textbf{Hodge Spectrum:} $\operatorname{Spec}(\Delta_p^C)$ (the multiset of eigenvalues of $\Delta_p^C$).
    \item \textbf{Spectral Gap:} $\lambda_1(\Delta_p^C)$, the smallest positive eigenvalue of $\Delta_p^C$ (if any).
    \item \textbf{Kirchhoff Index:} $Kf_p(\langle T \rangle) := \operatorname{Tr}\bigl((\Delta_p^C)^\dagger\bigr)$.
\end{enumerate}
\end{definition}

For $x,y \in C_p$ we can similarly define an \emph{effective resistance pseudometric}
\[
R_p^C(x,y) := \langle x-y, (\Delta_p^C)^\dagger(x-y)\rangle,
\]
where $(\Delta_p^C)^\dagger$ denotes the Moore–Penrose pseudoinverse. This is a genuine
metric on the orthogonal complement of $\ker(\Delta_p^C)$ and a pseudometric on $C_p$,
hence on $\langle T\rangle_p$.

\begin{remark}[Spectral invariants of generated submodules]
For a fixed tensor $T$, the isomorphism
\[
\langle T\rangle_\bullet \;\cong\; C_\bullet / K(T)_\bullet
\]
(Proposition~\ref{prop:isomorphism-classes-and-orbits}) shows that the projected
Laplacians $\Delta_p^C$ are determined functorially by the kernel sequence
$K(T)_\bullet$.
In particular, the Betti numbers $\beta_p(\langle T\rangle)$ and the spectra
$\operatorname{Spec}(\Delta_p^C)$ are invariants of the isomorphism class of
$\langle T\rangle_\bullet$, and hence of the point $\Psi(T)$ in the moduli space
$\mathcal M(\vec s)$ from Section~\ref{sec:generated}.
Traces of natural functions of $\Delta_p^C$ (for example the Foster-type sums
from Theorem~\ref{thm:foster}) therefore define algebraic functions on
$\mathcal M(\vec s)$.
For shapes such as $\vec s=(3,3)$ (Example~\ref{ex:shape_33}), these functions
vary non-trivially with the parameters $[v_{00}:v_{11}:v_{22}] \in \mathbb{P}^2$.
\end{remark}




\subsection{DSTM Foster Identity}

We derive a higher-dimensional analog of Foster's Reactance Theorem, relating diagonal resistances to the rank of the complex.

\begin{definition}[Diagonal Resistance]
Let $T_p = \partial_p^* (\partial_p^*)^\dagger$ be the orthogonal projection onto $\operatorname{im}(\partial_p^*)$. For a basis element $E_m \in X_p$, the \textbf{diagonal resistance} is
\[
r_{E_m} := \langle T_p E_m, E_m \rangle.
\]
\end{definition}

\begin{theorem}[DSTM Foster Identity]\label{thm:foster}
For the DSTM $X_\bullet(\vec{s}; K)$ and any degree $p \ge 1$:
\[
\sum_{m \in I_p} r_{E_m} = |I_{p-1}| - \operatorname{rank}(\partial_{p-1}^*).
\]
\end{theorem}

\begin{proof}
The sum of diagonal resistances is the trace of the projection $T_p$, which equals $\operatorname{rank}(\partial_p^*)$. By the rank-nullity theorem applied to the Hodge decomposition of $X_{p-1}$ (and using $H_{p-1}=0$):
\[
|I_{p-1}| = \operatorname{rank}(\partial_p) + \operatorname{rank}(\partial_{p-1}^*).
\]
Since $\operatorname{rank}(\partial_p) = \operatorname{rank}(\partial_p^*)$, we substitute to get $\operatorname{rank}(\partial_p^*) = |I_{p-1}| - \operatorname{rank}(\partial_{p-1}^*)$.
\end{proof}

\begin{corollary}
For $p=1$, since $\partial_0^*=0$, we have $\sum_{m \in I_1} r_{E_m} = |I_0|$.
For the constant shape $\vec{s}=(N, \dots, N)$, this sum is exactly 1.
\end{corollary}

\begin{proof}
For constant shape $\vec{s}=(N, \dots, N)$, we have $M_a(0)=0$ for all $a$
(cf.~Section~\ref{sec:combinatorics}). Thus $I_0 = \{ (0, \dots, 0) \}$ and
$|I_0|=1$.
\end{proof}

